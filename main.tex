\documentclass{article}
\usepackage[polish]{babel}
\usepackage[OT4]{fontenc}
\usepackage[utf8]{inputenc}
\usepackage{hyperref}

\title{\textbf{Praca projektowa z przedmiotu Sztuczna Inteligencja}}
\author{Bartłomiej Czajka 169522 2 EF-DI P1}
\date{Rzeszów, 2023}
\begin{document}
\maketitle
\pagebreak
dane nieuporządkowane, brak danych nieokreslonych,
24 kolumny z czego jedna odrzucamy calkiem (name)


\tableofcontents{}
\section{Opis projektu}
\subsection{Założenia projektowe}
Celem projektu jest realizacja sieci neuronowej uczonej za pomocą algorytmu sieci głębokiej, klasyfikującej chorobę Parkinsona oraz zbadanie wpływu parametrów sieci na proces uczenia.
Projekt został zrealizowany w języku Python z wykorzystaniem biblioteki PyTorch.
\subsection{Zestaw danych}
Zestaw danych uczących został pobrany ze strony \href{http://archive.ics.uci.edu/ml/datasets/Parkinsons}{http://archive.ics.uci.edu/ml/datasets/Parkinsons}.
Zawiera on 197 instancji, 23 cechy oraz 2 klasy.
Dokładniejszy opis cech zestawu:
\begin{itemize}
    \item name - ASCII subject name and recording number
    \item MDVP:Fo(Hz) - Average vocal fundamental frequency
    \item MDVP:Fhi(Hz) - Maximum vocal fundamental frequency
    \item MDVP:Flo(Hz) - Minimum vocal fundamental frequency
    \item MDVP:Jitter(\%), MDVP:Jitter(Abs), MDVP:RAP, MDVP:PPQ, Jitter:DDP - Several measures of variation in fundamental frequency
    \item MDVP:Shimmer, MDVP:Shimmer(dB), Shimmer:APQ3, Shimmer:APQ5, MDVP:APQ, Shimmer:DDA - Several measures of variation in amplitude
    \item NHR, HNR - Two measures of ratio of noise to tonal components in the voice
    \item status - Health status of the subject (one) - Parkinson's, (zero) - healthy
    \item RPDE, D2 - Two nonlinear dynamical complexity measures
    \item DFA - Signal fractal scaling exponent
    \item spread1, spread2, PPE - Three nonlinear measures of fundamental frequency variation
\end{itemize}
\subsection{Przygotowanie danych}
Sieć ma za zadanie sklasyfikować czy pacjent cierpi na chorobę Parkinsona.
\section{Wstęp teoretyczny}
Testuje spis treści :)

\end{document}