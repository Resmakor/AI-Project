\documentclass{article}
\usepackage[polish]{babel}
\usepackage[OT4]{fontenc}
\usepackage[utf8]{inputenc}
\usepackage{hyperref}

\title{\textbf{Praca projektowa z przedmiotu Sztuczna Inteligencja}}
\author{Bartłomiej Czajka 169522 2 EF-DI P1}
\date{Rzeszów, 2023}
\begin{document}
\maketitle
\pagebreak



\tableofcontents{}
\section{Opis projektu}
\subsection{Założenia projektowe}
Celem projektu jest realizacja sieci neuronowej uczonej za pomocą algorytmu sieci głębokiej, klasyfikującej chorobę Parkinsona oraz zbadanie wpływu parametrów sieci na proces uczenia.
Projekt został zrealizowany w języku Python z wykorzystaniem biblioteki PyTorch.
\subsection{Zestaw danych}
Zestaw danych uczących został pobrany ze strony \href{http://archive.ics.uci.edu/ml/datasets/Parkinsons}{http://archive.ics.uci.edu/ml/datasets/Parkinsons}.
Zawiera on 197 instancji, 23 cechy oraz 2 klasy. Dane są nieuporządkowane, nie ma danych nieokreślonych.
Dokładniejszy opis cech zestawu:
\begin{itemize}
    \item name - Nazwa badanego pacjenta w ASCII i numer nagrania.
    \item MDVP:Fo(Hz) - Średnia częstotliwość podstawowa głosu.
    \item MDVP:Fhi(Hz) - Maksymalna częstotliwość podstawowa głosu.
    \item MDVP:Flo(Hz) - Minimalna częstotliwość podstawowa głosu.
    \item MDVP:Jitter(\%), MDVP:Jitter(Abs), MDVP:RAP, MDVP:PPQ, Jitter:DDP - Kilka miar zmienności częstotliwości podstawowej.
    \item MDVP:Shimmer, MDVP:Shimmer(dB), Shimmer:APQ3, Shimmer:APQ5, MDVP:APQ, Shimmer:DDA - Kilka miar zmienności amplitudy.
    \item NHR, HNR - Dwie miary stosunku szumu do składowych tonalnych w głosie.
    \item status - Stan zdrowia badanego (jeden) - chory na Parkinsona, (zero) - zdrowy.
    \item RPDE, D2 - Dwie miary złożoności dynamicznej nieliniowej.
    \item DFA - Wykładnik skalowania fraktalnego sygnału.
    \item spread1, spread2, PPE - Trzy nieliniowe miary zmienności częstotliwości podstawowej.
\end{itemize}
\subsection{Przygotowanie danych}
Sieć ma za zadanie sklasyfikować czy pacjent cierpi na chorobę Parkinsona.
\section{Wstęp teoretyczny}
Testuje spis treści :)

\end{document}